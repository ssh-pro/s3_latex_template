%This \LaTeX\ template is released into the public domain by the copyright holders.
\documentclass[10.5pt,twocolumn]{jsarticle}
\usepackage{sounin}

\usepackage{graphicx}
\usepackage[dvipdfmx]{color}
\usepackage{url}
\pagestyle{empty}

\title{\fontsize{14pt}{0cm}\gt
高三総合人間科 \LaTeX テンプレートファイル}
\author{\fontsize{12pt}{0cm}\mc
名大附 花子}
\date{}


\begin{document}
\begin{abstract}
各段落の最初の一文をつなぐと、要約文ができあがるようにする。現在についてという点で関心が集まっている。先行研究ではが明らかになっている一方でについてはまだわかっていない。そこで本研究においてを研究課題とし、という仮説の下に、を研究した。その結果が明らかになった。本研究にはに対し、という意義があり、今後が期待される。
\end{abstract}
\maketitle

\part{序論 研究課題について}
\section{研究の背景}
現在についてという点で関心が集まっている。先行研究ではが明らかになっている一方でについてはまだわかっていない。そこで本研究においてを研究課題とし、という仮説の下に、を研究した。その結果が明らかになった。本研究にはに対し、という意義があり、今後が期待される。

\section{先行研究}
現在についてという点で関心が集まっている。
Aとはである。
Aに関心が集まったのは頃である。その後(時系列順に経緯を書く)、Bという点で注目されるようになった。

\section{研究課題と仮説}
本研究では〜を明らかにしていく。
先行研究等の分析から〜という仮説を立てた。

\section{研究の目的と意義}

\part{本論 調査結果と考察}
\section{調査テーマ}

\section{調査時期と方法}

\section{調査結果}

\section{考察}

\part{結論}
\section{結論}

\section{今後の展望}
\begin{thebibliography}{2501}

\end{thebibliography}

\end{document}
